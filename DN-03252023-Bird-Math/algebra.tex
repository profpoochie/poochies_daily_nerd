
\documentclass{article}
%%%%%%%%%%%%%%%%%%%%%%%%%%%%%%%%%%%%%%%%%%%%%%%%%%%%%%%%%%%%%%%%%%%%%%%%%%%%%%%%%%%%%%%%%%%%%%%%%%%%%%%%%%%%%%%%%%%%%%%%%%%%%%%%%%%%%%%%%%%%%%%%%%%%%%%%%%%%%%%%%%%%%%%%%%%%%%%%%%%%%%%%%%%%%%%%%%%%%%%%%%%%%%%%%%%%%%%%%%%%%%%%%%%%%%%%%%%%%%%%%%%%%%%%%%%%
\usepackage{verbatim}

%TCIDATA{OutputFilter=LATEX.DLL}
%TCIDATA{Version=5.50.0.2953}
%TCIDATA{<META NAME="SaveForMode" CONTENT="1">}
%TCIDATA{BibliographyScheme=Manual}
%TCIDATA{Created=Saturday, March 25, 2023 22:57:12}
%TCIDATA{LastRevised=Sunday, March 26, 2023 01:47:11}
%TCIDATA{<META NAME="GraphicsSave" CONTENT="32">}
%TCIDATA{<META NAME="DocumentShell" CONTENT="Standard LaTeX\Blank - Standard LaTeX Article">}
%TCIDATA{CSTFile=40 LaTeX article.cst}

\newtheorem{theorem}{Theorem}
\newtheorem{acknowledgement}[theorem]{Acknowledgement}
\newtheorem{algorithm}[theorem]{Algorithm}
\newtheorem{axiom}[theorem]{Axiom}
\newtheorem{case}[theorem]{Case}
\newtheorem{claim}[theorem]{Claim}
\newtheorem{conclusion}[theorem]{Conclusion}
\newtheorem{condition}[theorem]{Condition}
\newtheorem{conjecture}[theorem]{Conjecture}
\newtheorem{corollary}[theorem]{Corollary}
\newtheorem{criterion}[theorem]{Criterion}
\newtheorem{definition}[theorem]{Definition}
\newtheorem{example}[theorem]{Example}
\newtheorem{exercise}[theorem]{Exercise}
\newtheorem{lemma}[theorem]{Lemma}
\newtheorem{notation}[theorem]{Notation}
\newtheorem{problem}[theorem]{Problem}
\newtheorem{proposition}[theorem]{Proposition}
\newtheorem{remark}[theorem]{Remark}
\newtheorem{solution}[theorem]{Solution}
\newtheorem{summary}[theorem]{Summary}
\newenvironment{proof}[1][Proof]{\noindent\textbf{#1.} }{\ \rule{0.5em}{0.5em}}
\input{tcilatex}
\begin{document}


\part{Algebra}

\section{1.1 Introduction}

Introduction and practice to algebra

\section{1.2 \ Revision of basic laws}

\subsection{(a) Basic operations and laws of indices:}

The law of indices are:

(i) $a^{m}\times a^{n}=a^{m+n}$

\begin{enumerate}
\item $a^{m}\times a^{n}=a^{m+n}$

\item $\dfrac{a^{m}}{a^{n}}=a^{m-n}$

\item $\left( a^{m}\right) ^{n}=a^{m\times n}=a^{mn}$

\item $a^{\dfrac{m}{n}}=\sqrt[n]{a^{m}}$

\item $a^{-n}=\dfrac{1}{a^{n}}$

\item $a^{0}=1$
\end{enumerate}

\begin{problem}
1. Evaluate $4a^{2}bc^{3}-2ac$ when $a=2$, $b=\dfrac{1}{2}$ , $c=1\dfrac{1}{2%
}$

\begin{solution}
Mathematica code:
\end{solution}
\end{problem}

\begin{verbatim}
In[]:= 4 a^2 b c^3 - 2 a c /. {a -> 2, b -> 1/2, c -> 1.5}
Out[]= 21.
 
\end{verbatim}

Problem 2. Multiply $3x+2y$ by $x-y$

Solution:

Mathematica code:
\begin{verbatim}
In[]:= Expand[(3 x + 2 y) (x - y)] // TraditionalForm
Out[]= 3 x^2 - x y - 2 y^2
\end{verbatim}

Answer is $3x^{2}-xy-2y^{2}$

\bigskip 

Problem 3. Simplify $\dfrac{a^{3}b^{2}c^{4}}{abc^{-2}}$ and evaluate when $%
a=3$, $b=\dfrac{1}{8}$ and $c=2$.

Solution:

Mathematica code:
\begin{verbatim}
In[]:= Expand[(3 x + 2 y) (x - y)] // TraditionalForm
Out[]= 3 x^2 - x y - 2 y^2
In[]:= Simplify[(a^3 b^2 c^4)/(a b c^-2)]
Out[]= a^2 b c^6
In[]:= % /. {a -> 3, b -> 1/8, c -> 2}
Out[]= 72
 
\end{verbatim}

Problem 4. Simplify $\dfrac{x^{2}y^{3}+xy^{2}}{xy}$.

Solution:

Mathematica code:
\begin{verbatim}
In[]:= Simplify[(x^2 y^3 + x y^2)/(x y)]
Out[]= y (1 + x y)
\end{verbatim}

The answer is $y\left( xy+1\right) $
\begin{verbatim}
 
 
In[]:= 
Out[]= 
\end{verbatim}

\end{document}
