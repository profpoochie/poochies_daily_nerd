
\documentclass{article}
%%%%%%%%%%%%%%%%%%%%%%%%%%%%%%%%%%%%%%%%%%%%%%%%%%%%%%%%%%%%%%%%%%%%%%%%%%%%%%%%%%%%%%%%%%%%%%%%%%%%%%%%%%%%%%%%%%%%%%%%%%%%%%%%%%%%%%%%%%%%%%%%%%%%%%%%%%%%%%%%%%%%%%%%%%%%%%%%%%%%%%%%%%%%%%%%%%%%%%%%%%%%%%%%%%%%%%%%%%%%%%%%%%%%%%%%%%%%%%%%%%%%%%%%%%%%
\usepackage{amssymb}

%TCIDATA{OutputFilter=LATEX.DLL}
%TCIDATA{Version=5.50.0.2953}
%TCIDATA{<META NAME="SaveForMode" CONTENT="1">}
%TCIDATA{BibliographyScheme=Manual}
%TCIDATA{Created=Sunday, January 05, 2020 14:53:20}
%TCIDATA{LastRevised=Wednesday, January 15, 2020 05:37:44}
%TCIDATA{<META NAME="GraphicsSave" CONTENT="32">}
%TCIDATA{<META NAME="DocumentShell" CONTENT="Standard LaTeX\Blank - Standard LaTeX Article">}
%TCIDATA{CSTFile=40 LaTeX article.cst}

\newtheorem{theorem}{Theorem}
\newtheorem{acknowledgement}[theorem]{Acknowledgement}
\newtheorem{algorithm}[theorem]{Algorithm}
\newtheorem{axiom}[theorem]{Axiom}
\newtheorem{case}[theorem]{Case}
\newtheorem{claim}[theorem]{Claim}
\newtheorem{conclusion}[theorem]{Conclusion}
\newtheorem{condition}[theorem]{Condition}
\newtheorem{conjecture}[theorem]{Conjecture}
\newtheorem{corollary}[theorem]{Corollary}
\newtheorem{criterion}[theorem]{Criterion}
\newtheorem{definition}[theorem]{Definition}
\newtheorem{example}[theorem]{Example}
\newtheorem{exercise}[theorem]{Exercise}
\newtheorem{lemma}[theorem]{Lemma}
\newtheorem{notation}[theorem]{Notation}
\newtheorem{problem}[theorem]{Problem}
\newtheorem{proposition}[theorem]{Proposition}
\newtheorem{remark}[theorem]{Remark}
\newtheorem{solution}[theorem]{Solution}
\newtheorem{summary}[theorem]{Summary}
\newenvironment{proof}[1][Proof]{\noindent\textbf{#1.} }{\ \rule{0.5em}{0.5em}}
\input{tcilatex}
\begin{document}


\section{\protect\bigskip Chapter 2: Coordinate Systems and Transformation}

\subsection{2.1. Introduction}

An orthogonal system is one in which the coordinates are mutually
perpendicular.

\bigskip

\subsection{2.2. Cartesian Coordinates $(x,y,z)$}

The ranges of coordinate system

\begin{eqnarray*}
-\infty &<&x<\infty \\
-\infty &<&y<\infty \\
-\infty &<&z<\infty
\end{eqnarray*}

A vector $\mathbf{A}$ in Cartesian (otherwise known as rectangular)
coordinates can be written as 
\[
\left( A_{x},A_{y},A_{z}\right) \text{ ~~or ~~~}A_{x}\mathbf{a}_{x}+A_{y}%
\mathbf{a}_{y}+A_{z}\mathbf{a}_{z} 
\]%
where $\mathbf{a}_{x},$ $\mathbf{a}_{x},$ $\mathbf{a}_{x}$ are units vectors
along the $x-,$ $y-,$ and $z-$ direction as shown.

\bigskip

\subsection{2.3. Circular Cylindrical Coordinate $(\protect\rho ,\protect%
\phi ,z)$}

The ranges of coordinate system

\begin{eqnarray*}
0\, &\leq &\rho <\infty \\
0\, &\leq &\phi <\infty \\
-\infty &<&z<\infty
\end{eqnarray*}

A vector $\mathbf{A}$ in cylindrical coodinates can be written as 
\[
\left( A_{\rho },A_{\phi },A_{z}\right) \text{ ~~or ~~~}A_{\rho }\mathbf{a}%
_{\rho }+A_{\phi }\mathbf{a}_{\phi }+A_{z}\mathbf{a}_{z} 
\]

The magnitude of $\mathbf{A}$ is 
\[
\left\vert \mathbf{A}\right\vert =\left( A_{\rho }^{2}+A_{\phi
}^{2}+A_{z}^{2}\right) 
\]

The relation of unit vectors are 
\begin{eqnarray*}
\mathbf{a}_{\rho }\bullet \mathbf{a}_{\rho } &=&\mathbf{a}_{\phi }\bullet 
\mathbf{a}_{\phi }=\mathbf{a}_{z}\bullet \mathbf{a}_{z}=1 \\
\mathbf{a}_{\rho }\bullet \mathbf{a}_{\phi } &=&\mathbf{a}_{\phi }\bullet 
\mathbf{a}_{z}=\mathbf{a}_{z}\bullet \mathbf{a}_{\rho }=0 \\
\mathbf{a}_{\rho }\times \mathbf{a}_{\phi } &=&\mathbf{a}_{z} \\
\mathbf{a}_{\phi }\times \mathbf{a}_{z} &=&\mathbf{a}_{\rho } \\
\mathbf{a}_{z}\times \mathbf{a}_{\rho } &=&\mathbf{a}_{\phi }
\end{eqnarray*}

The relationships between the variables $\left( x,y,z\right) $ of the
Cartesian coordinate system and those of the cylindrical system and those of
the cylindrical system $(\rho ,\phi ,z)$ 
\begin{eqnarray*}
\rho &=&\sqrt{x^{2}+y^{2}} \\
\phi &=&\arctan \left( \dfrac{y}{x}\right) \\
z &=&z
\end{eqnarray*}

or 
\begin{eqnarray*}
x &=&\rho \cos \left( \phi \right) \\
y &=&\rho \sin \left( \phi \right) \\
z &=&z
\end{eqnarray*}

The relationship between $\left( \mathbf{a}_{x},\mathbf{a}_{y},\mathbf{a}%
_{z}\right) $ and $\left( \mathbf{a}_{\rho },\mathbf{a}_{\phi },\mathbf{a}%
_{z}\right) $ are obtained geometrically 
\begin{eqnarray*}
\mathbf{a}_{x} &=&\cos \left( \phi \right) \mathbf{a}_{\rho }-\sin \left(
\phi \right) \mathbf{a}_{\phi } \\
\mathbf{a}_{y} &=&\sin \left( \phi \right) \mathbf{a}_{\rho }+\cos \left(
\phi \right) \mathbf{a}_{\phi } \\
\mathbf{a}_{z} &=&\mathbf{a}_{z}
\end{eqnarray*}

or 
\begin{eqnarray*}
\mathbf{a}_{\rho } &=&\cos \left( \phi \right) \mathbf{a}_{x}+\sin \left(
\phi \right) \mathbf{a}_{y} \\
\mathbf{a}_{\phi } &=&-\sin \left( \phi \right) \mathbf{a}_{x}+\cos \left(
\phi \right) \mathbf{a}_{y} \\
\mathbf{a}_{z} &=&\mathbf{a}_{z}
\end{eqnarray*}

The relationship between $\left( A_{x},A_{y},A_{z}\right) $ and $\left(
A_{\rho },A_{\phi },A_{z}\right) $ are obtained by substitution%
\[
\mathbf{A=}\left( A_{x}\cos \left( \phi \right) +A_{y}\sin \left( \phi
\right) \right) \mathbf{a}_{\rho }+\left( -A_{x}\sin \left( \phi \right)
+A_{y}\cos \left( \phi \right) \right) +A_{z}\mathbf{a}_{z} 
\]

or 
\begin{eqnarray*}
A_{\rho } &=&A_{x}\cos \left( \phi \right) +A_{y}\sin \left( \phi \right) \\
A_{\phi } &=&-A_{x}\sin \left( \phi \right) +A_{y}\cos \left( \phi \right) \\
A_{z} &=&A_{z}
\end{eqnarray*}

In matrix form, transformation of vector $\mathbf{A}$ form $\left(
A_{x},A_{y},A_{z}\right) \longrightarrow \left( A_{\rho },A_{\phi
},A_{z}\right) $ 
\[
\left[ 
\begin{array}{c}
A_{p} \\ 
A_{\phi } \\ 
A_{z}%
\end{array}%
\right] =\left[ 
\begin{array}{ccc}
\cos \left( \phi \right) & \sin \left( \phi \right) & 0 \\ 
-\sin \left( \phi \right) & \cos \left( \phi \right) & 0 \\ 
0 & 0 & 1%
\end{array}%
\right] \left[ 
\begin{array}{c}
A_{x} \\ 
A_{y} \\ 
A_{z}%
\end{array}%
\right] 
\]

By matrix inversion, transformation of vector $\mathbf{A}$ form $\left(
A_{\rho },A_{\phi },A_{z}\right) $ $\longrightarrow \left(
A_{x},A_{y},A_{z}\right) $ 
\[
\left[ 
\begin{array}{c}
A_{x} \\ 
A_{y} \\ 
A_{z}%
\end{array}%
\right] =\left[ 
\begin{array}{ccc}
\cos \phi & -\sin \phi & 0 \\ 
\sin \phi & \cos \phi & 0 \\ 
0 & 0 & 1%
\end{array}%
\right] \left[ 
\begin{array}{c}
A_{p} \\ 
A_{\phi } \\ 
A_{z}%
\end{array}%
\right] 
\]

Alternatively, using dot product%
\[
\left[ 
\begin{array}{c}
A_{x} \\ 
A_{y} \\ 
A_{z}%
\end{array}%
\right] =\left[ 
\begin{array}{ccc}
\mathbf{a}_{x}\bullet \mathbf{a}_{\rho } & \mathbf{a}_{x}\bullet \mathbf{a}%
_{\phi } & \mathbf{a}_{x}\bullet \mathbf{a}_{z} \\ 
\mathbf{a}_{y}\bullet \mathbf{a}_{p} & \mathbf{a}_{y}\bullet \mathbf{a}%
_{\phi } & \mathbf{a}_{y}\bullet \mathbf{a}_{z} \\ 
\mathbf{a}_{z}\bullet \mathbf{a}_{p} & \mathbf{a}_{z}\bullet \mathbf{a}%
_{\phi } & \mathbf{a}_{z}\bullet \mathbf{a}_{z}%
\end{array}%
\right] \left[ 
\begin{array}{c}
A_{p} \\ 
A_{\phi } \\ 
A_{z}%
\end{array}%
\right] 
\]

\subsection{2.4. Spherical Coordinates $\left( r,\protect\theta ,\protect%
\phi \right) $}

The range of variables 
\begin{eqnarray*}
0 &\leq &r<\infty \\
0 &\leq &\theta \leq \pi \\
0 &\leq &\phi <2\pi
\end{eqnarray*}

A vector $\mathbf{A}$ in spherical coordinates may be written as 
\[
\left( A_{r},A_{\theta },A_{\phi }\right) ~~~~or~~~~A_{r}\mathbf{a}%
_{r}+A_{\theta }\mathbf{a}_{\theta }+A_{\phi }\mathbf{a}_{\phi } 
\]

The magniture of $\mathbf{A}$ is 
\[
\left\vert \mathbf{A}\right\vert =\sqrt{A_{r}^{2}+A_{\theta }^{2}+A_{\phi
}^{2}} 
\]

The unit vector relationships 
\begin{eqnarray*}
\mathbf{a}_{r}\bullet \mathbf{a}_{r} &=&\mathbf{a}_{\theta }\bullet \mathbf{a%
}_{\theta }=\mathbf{a}_{\phi }\bullet \mathbf{a}_{\phi }=1 \\
\mathbf{a}_{r}\bullet \mathbf{a}_{\theta } &=&\mathbf{a}_{\theta }\bullet 
\mathbf{a}_{\phi }=\mathbf{a}_{\phi }\bullet \mathbf{a}_{r}=0 \\
\mathbf{a}_{r}\times \mathbf{a}_{\theta } &=&\mathbf{a}_{\phi } \\
\mathbf{a}_{\theta }\times \mathbf{a}_{\phi } &=&\mathbf{a}_{r} \\
\mathbf{a}_{\phi }\times \mathbf{a}_{r} &=&\mathbf{a}_{\theta }
\end{eqnarray*}

The space variables $\left( x,y,z\right) $ in Cartesian coordinates can be
related to variables $\left( r,\theta ,\phi \right) $ of a spherical
coordinate system 
\begin{eqnarray*}
r &=&\sqrt{x^{2}+y^{2}+z^{2}} \\
\theta &=&\arctan \left( \dfrac{\sqrt{x^{2}+y^{2}}}{z}\right) \\
\phi &=&\arctan \left( \dfrac{y}{x}\right)
\end{eqnarray*}

or 
\begin{eqnarray*}
x &=&r\sin \left( \theta \right) \cos \left( \phi \right) \\
y &=&r\sin \left( \theta \right) \sin \left( \phi \right) \\
z &=&r\cos \left( \theta \right)
\end{eqnarray*}

The unit vectors $\mathbf{a}_{x},\mathbf{a}_{y},\mathbf{a}_{z}$ and $\mathbf{%
a}_{r},\mathbf{a}_{\theta },\mathbf{a}_{\phi }$ are related as follows:

$\mathbf{a}_{r},\mathbf{a}_{\theta },\mathbf{a}_{\phi }\longrightarrow 
\mathbf{a}_{x},\mathbf{a}_{y},\mathbf{a}_{z}$ 
\begin{eqnarray*}
\mathbf{a}_{x} &=&\sin \left( \theta \right) \cos \left( \phi \right) 
\mathbf{a}_{r}+\cos \left( \theta \right) \cos \left( \phi \right) \mathbf{a}%
_{\theta }-\sin \left( \phi \right) \mathbf{a}_{\phi } \\
\mathbf{a}_{y} &=&\sin \left( \theta \right) \sin \left( \phi \right) 
\mathbf{a}_{r}+\cos \left( \theta \right) \sin \left( \phi \right) \mathbf{a}%
_{\theta }+\cos \left( \phi \right) \mathbf{a}_{\phi } \\
\mathbf{a}_{z} &=&\cos \left( \theta \right) \mathbf{a}_{r}-\sin \left(
\theta \right) \mathbf{a}_{\theta }
\end{eqnarray*}

or $\mathbf{a}_{x},\mathbf{a}_{y},\mathbf{a}_{z}\longrightarrow \mathbf{a}%
_{r},\mathbf{a}_{\theta },\mathbf{a}_{\phi }$ 
\begin{eqnarray*}
\mathbf{a}_{r} &=&\sin \left( \theta \right) \cos \left( \phi \right) 
\mathbf{a}_{x}+\sin \left( \theta \right) \sin \left( \phi \right) \mathbf{a}%
_{y}+\cos \left( \theta \right) \mathbf{a}_{z} \\
\mathbf{a}_{\theta } &=&\cos \left( \theta \right) \sin \left( \phi \right) 
\mathbf{a}_{x}+\cos \left( \theta \right) \sin \left( \phi \right) \mathbf{a}%
_{y}-\cos \left( \theta \right) \mathbf{a}_{z} \\
\mathbf{a}_{\phi } &=&-\sin \left( \phi \right) \mathbf{a}_{x}+\cos \left(
\theta \right) \mathbf{a}_{y}
\end{eqnarray*}

The components of vector $\mathbf{A=}\left( A_{x},A_{y},A_{z}\right) $ and $%
\mathbf{A=}\left( A_{r},A_{\theta },A_{\phi }\right) $ are related by the
equation 
\[
\mathbf{A=}\left( A_{x}\sin \left( \theta \right) \cos \left( \phi \right)
+A_{y}\sin \left( \theta \right) \sin \left( \phi \right) +A_{z}\cos \left(
\theta \right) \right) \mathbf{a}_{r}+\left( A_{x}\cos \left( \theta \right)
\cos \left( \phi \right) +A_{y}\cos \left( \theta \right) \sin \left( \phi
\right) -A_{z}\sin \left( \theta \right) \right) \mathbf{a}_{\theta }-\left(
-A_{x}\sin \left( \phi \right) +A_{y}\cos \left( \phi \right) \right) 
\mathbf{a}_{\phi } 
\]

or 
\begin{eqnarray*}
A_{r} &=&A_{x}\sin \left( \theta \right) \cos \left( \phi \right) +A_{y}\sin
\left( \theta \right) \sin \left( \phi \right) +A_{z}\cos \left( \theta
\right) \\
A_{\theta } &=&A_{x}\cos \left( \theta \right) \cos \left( \phi \right)
+A_{y}\cos \left( \theta \right) \sin \left( \phi \right) -A_{z}\sin \left(
\theta \right) \\
A_{\phi } &=&-A_{x}\sin \left( \phi \right) +A_{y}\cos \left( \phi \right)
\end{eqnarray*}

In matrix form, $\left( A_{x},A_{y},A_{z}\right) \longrightarrow \left(
A_{r},A_{\theta },A_{\phi }\right) $ 
\[
\left[ 
\begin{array}{c}
A_{r} \\ 
A_{\theta } \\ 
A_{\phi }%
\end{array}%
\right] =\left[ 
\begin{array}{ccc}
\sin \left( \theta \right) \cos \left( \phi \right) & \sin \left( \theta
\right) \sin \left( \phi \right) & \cos \left( \theta \right) \\ 
\cos \left( \theta \right) \cos \left( \phi \right) & \cos \left( \theta
\right) \sin \left( \phi \right) & -\sin \left( \theta \right) \\ 
-\sin \left( \phi \right) & \cos \left( \phi \right) & 0%
\end{array}%
\right] \left[ 
\begin{array}{c}
A_{x} \\ 
A_{y} \\ 
A_{z}%
\end{array}%
\right] 
\]%
$\allowbreak $

By matrix inversion, transformation of vector $\mathbf{A}$ form $\left(
A_{r},A_{\theta },A_{\phi }\right) $ $\longrightarrow \left(
A_{x},A_{y},A_{z}\right) $ 
\[
\left[ 
\begin{array}{c}
A_{x} \\ 
A_{y} \\ 
A_{z}%
\end{array}%
\right] =\left[ 
\begin{array}{ccc}
\sin \left( \theta \right) \cos \left( \phi \right) & \cos \left( \theta
\right) \cos \left( \phi \right) & -\sin \left( \phi \right) \\ 
\sin \left( \theta \right) \sin \left( \phi \right) & \cos \left( \theta
\right) \sin \left( \phi \right) & \cos \left( \phi \right) \\ 
\cos \left( \theta \right) & -\sin \left( \theta \right) & 0%
\end{array}%
\right] \left[ 
\begin{array}{c}
A_{r} \\ 
A_{\theta } \\ 
A_{\phi }%
\end{array}%
\right] 
\]

Alternatively, using dot product 
\[
\left[ 
\begin{array}{c}
A_{r} \\ 
A_{\theta } \\ 
A_{\phi }%
\end{array}%
\right] =\left[ 
\begin{array}{ccc}
\mathbf{a}_{r}\bullet \mathbf{a}_{x} & \mathbf{a}_{r}\bullet \mathbf{a}_{y}
& \mathbf{a}_{r}\bullet \mathbf{a}_{z} \\ 
\mathbf{a}_{\theta }\bullet \mathbf{a}_{x} & \mathbf{a}_{\theta }\bullet 
\mathbf{a}_{y} & \mathbf{a}_{\theta }\bullet \mathbf{a}_{z} \\ 
\mathbf{a}_{\phi }\bullet \mathbf{a}_{x} & \mathbf{a}_{\phi }\bullet \mathbf{%
a}_{y} & \mathbf{a}_{\phi }\bullet \mathbf{a}_{z}%
\end{array}%
\right] \left[ 
\begin{array}{c}
A_{x} \\ 
A_{y} \\ 
A_{z}%
\end{array}%
\right] 
\]

The distance between two points is necessary in EM theory. The distance $d$
between two points with position vector $\mathbf{r}_{1}$ and $\mathbf{r}_{2}$
is generally given by 
\[
d=\left\vert \mathbf{r}_{2}-\mathbf{r}_{1}\right\vert 
\]

or 
\begin{eqnarray*}
d^{2} &=&\left( x_{2}-x_{1}\right) ^{2}+\left( y_{2}-y_{1}\right)
^{2}+\left( z_{2}-z_{1}\right) ^{2}~~~~\text{(Cartesian)} \\
d^{2} &=&\rho _{2}^{2}+\rho _{1}^{2}-2\rho _{1}\rho _{2}\cos \left( \phi
_{2}-\phi _{1}\right) +\left( z_{2}-z_{1}\right) ^{2}~\text{~~~(cylindrical)}
\\
d^{2} &=&r_{2}^{2}+r_{1}^{2}-2r_{1}r_{2}\cos \left( \theta _{2}\right) \cos
\left( \theta _{1}\right) -2r_{1}r_{2}\sin \left( \theta _{2}\right) \sin
\left( \theta _{1}\right) \cos \left( \phi _{2}-\phi _{1}\right) ~~~~\text{%
(spherical)}
\end{eqnarray*}

\bigskip

\subsubsection{Example 2.1}

Given point $P\left( -2,6,3\right) $ and vector $\mathbf{A=}y\mathbf{a}%
_{x}+\left( x+z\right) \mathbf{a}_{y}$, express $P$ and $\mathbf{A}$ in
cylindrical and spherical coordinates. Evaluate $\mathbf{A}$ at $P$ in the
Cartesian, cylindrical, and spherical systems.

Solution:

$\because $ $P\left( -2,6,3\right) $ $\therefore $ $x=-2;$ $y=6;$ and $z=3$.
Using the formula to get $\left( \rho ,\phi ,z\right) $ 
\begin{eqnarray*}
\rho &=&\sqrt{x^{2}+y^{2}}=\sqrt{\left( -2\right) ^{2}+6^{2}}=2\sqrt{10}%
=6.\,\allowbreak 324\,6 \\
\phi &=&\arctan \left( \dfrac{y}{x}\right) =\arctan \left( \dfrac{6}{-2}%
\right) =\left( 180\unit{%
%TCIMACRO{\U{b0}}%
%BeginExpansion
{{}^\circ}%
%EndExpansion
}-71.\,\allowbreak 565\unit{%
%TCIMACRO{\U{b0}}%
%BeginExpansion
{{}^\circ}%
%EndExpansion
}\right) =108.\,\allowbreak 43\unit{%
%TCIMACRO{\U{b0}}%
%BeginExpansion
{{}^\circ}%
%EndExpansion
} \\
z &=&6
\end{eqnarray*}

$\therefore P\left( -2,6,3\right) _{Cartesian}=P\left( 6.\,\allowbreak
324\,6,108.\,\allowbreak 43\unit{%
%TCIMACRO{\U{b0}}%
%BeginExpansion
{{}^\circ}%
%EndExpansion
},3\right) _{cylindrical}$

$\because $ $\mathbf{A=}y\mathbf{a}_{x}+\left( x+z\right) \mathbf{a}_{y}=6%
\mathbf{a}_{x}+\left( -2+3\right) \mathbf{a}_{y}=\allowbreak 6\mathbf{a}_{x}+%
\mathbf{a}_{y},$ $\therefore A_{x}=6;$ $A_{y}=1$ and $A_{z}=0$

Using the matrix conversion $\left( A_{x},A_{y},A_{z}\right) \longrightarrow
\left( A_{\rho },A_{\phi },A_{z}\right) $ 
\begin{eqnarray*}
\left[ 
\begin{array}{c}
A_{p} \\ 
A_{\phi } \\ 
A_{z}%
\end{array}%
\right] &=&\left[ 
\begin{array}{ccc}
\cos \left( \phi \right) & \sin \left( \phi \right) & 0 \\ 
-\sin \left( \phi \right) & \cos \left( \phi \right) & 0 \\ 
0 & 0 & 1%
\end{array}%
\right] \left[ 
\begin{array}{c}
A_{x} \\ 
A_{y} \\ 
A_{z}%
\end{array}%
\right] \\
&=&\left[ 
\begin{array}{ccc}
\cos \left( 108.\,\allowbreak 43\unit{%
%TCIMACRO{\U{b0}}%
%BeginExpansion
{{}^\circ}%
%EndExpansion
}\right) & \sin \left( 108.\,\allowbreak 43\unit{%
%TCIMACRO{\U{b0}}%
%BeginExpansion
{{}^\circ}%
%EndExpansion
}\right) & 0 \\ 
-\sin \left( 108.\,\allowbreak 43\unit{%
%TCIMACRO{\U{b0}}%
%BeginExpansion
{{}^\circ}%
%EndExpansion
}\right) & \cos \left( 108.\,\allowbreak 43\unit{%
%TCIMACRO{\U{b0}}%
%BeginExpansion
{{}^\circ}%
%EndExpansion
}\right) & 0 \\ 
0 & 0 & 1%
\end{array}%
\right] \left[ 
\begin{array}{c}
6 \\ 
1 \\ 
0%
\end{array}%
\right] \\
\left[ 
\begin{array}{c}
A_{p} \\ 
A_{\phi } \\ 
A_{z}%
\end{array}%
\right] &=&\left[ 
\begin{array}{c}
-0.948\,16 \\ 
-6.\,\allowbreak 008\,4 \\ 
0%
\end{array}%
\right] \allowbreak
\end{eqnarray*}

$\therefore \mathbf{A=}-0.948\,16\mathbf{a}_{\rho }+-6.\,\allowbreak 008\,4\,%
\mathbf{a}_{\phi }$ ~(cylindrical coordinates)

$\because $ $P\left( -2,6,3\right) $ $\therefore $ $x=-2;$ $y=6;$ and $z=3$.
Using the formula to get $\left( r,\theta ,\phi \right) $ 
\begin{eqnarray*}
r &=&\sqrt{x^{2}+y^{2}+z^{2}}=\sqrt{\left( -2\right) ^{2}+6^{2}+3^{2}}=7 \\
\theta &=&\arctan \left( \dfrac{\sqrt{x^{2}+y^{2}}}{z}\right) =\arctan
\left( \dfrac{\sqrt{\left( -2\right) ^{2}+6^{2}}}{3}\right)
=64.\,\allowbreak 624\unit{%
%TCIMACRO{\U{b0}}%
%BeginExpansion
{{}^\circ}%
%EndExpansion
} \\
\phi &=&\arctan \left( \dfrac{y}{x}\right) =\arctan \left( \dfrac{6}{-2}%
\right) =\left( 180\unit{%
%TCIMACRO{\U{b0}}%
%BeginExpansion
{{}^\circ}%
%EndExpansion
}-71.\,\allowbreak 565\unit{%
%TCIMACRO{\U{b0}}%
%BeginExpansion
{{}^\circ}%
%EndExpansion
}\right) =108.\,\allowbreak 43\unit{%
%TCIMACRO{\U{b0}}%
%BeginExpansion
{{}^\circ}%
%EndExpansion
}
\end{eqnarray*}

$\therefore $ $P\left( -2,6,3\right) _{Cartesian}=P\left( 7,64.\,\allowbreak
624\unit{%
%TCIMACRO{\U{b0}}%
%BeginExpansion
{{}^\circ}%
%EndExpansion
},108.\,\allowbreak 43\unit{%
%TCIMACRO{\U{b0}}%
%BeginExpansion
{{}^\circ}%
%EndExpansion
}\right) _{spherical}$

$\because $ $\mathbf{A=}y\mathbf{a}_{x}+\left( x+z\right) \mathbf{a}_{y}=6%
\mathbf{a}_{x}+\left( -2+3\right) \mathbf{a}_{y}=\allowbreak 6\mathbf{a}_{x}+%
\mathbf{a}_{y},$ $\therefore A_{x}=6;$ $A_{y}=1$ and $A_{z}=0$

Using the matrix conversion $\left( A_{x},A_{y},A_{z}\right) \longrightarrow
\left( A_{r},A_{\theta },A_{\phi }\right) $ 
\begin{eqnarray*}
\left[ 
\begin{array}{c}
A_{r} \\ 
A_{\theta } \\ 
A_{\phi }%
\end{array}%
\right] &=&\left[ 
\begin{array}{ccc}
\sin \left( \theta \right) \cos \left( \phi \right) & \sin \left( \theta
\right) \sin \left( \phi \right) & \cos \left( \theta \right) \\ 
\cos \left( \theta \right) \cos \left( \phi \right) & \cos \left( \theta
\right) \sin \left( \phi \right) & -\sin \left( \theta \right) \\ 
-\sin \left( \phi \right) & \cos \left( \phi \right) & 0%
\end{array}%
\right] \left[ 
\begin{array}{c}
A_{x} \\ 
A_{y} \\ 
A_{z}%
\end{array}%
\right] \\
&=&\left[ 
\begin{array}{ccc}
\sin \left( 64.\,\allowbreak 624\unit{%
%TCIMACRO{\U{b0}}%
%BeginExpansion
{{}^\circ}%
%EndExpansion
}\right) \cos \left( 108.\,\allowbreak 43\unit{%
%TCIMACRO{\U{b0}}%
%BeginExpansion
{{}^\circ}%
%EndExpansion
}\right) & \sin \left( 64.\,\allowbreak 624\unit{%
%TCIMACRO{\U{b0}}%
%BeginExpansion
{{}^\circ}%
%EndExpansion
}\right) \sin \left( 108.\,\allowbreak 43\unit{%
%TCIMACRO{\U{b0}}%
%BeginExpansion
{{}^\circ}%
%EndExpansion
}\right) & \cos \left( 64.\,\allowbreak 624\unit{%
%TCIMACRO{\U{b0}}%
%BeginExpansion
{{}^\circ}%
%EndExpansion
}\right) \\ 
\cos \left( 64.\,\allowbreak 624\unit{%
%TCIMACRO{\U{b0}}%
%BeginExpansion
{{}^\circ}%
%EndExpansion
}\right) \cos \left( 108.\,\allowbreak 43\unit{%
%TCIMACRO{\U{b0}}%
%BeginExpansion
{{}^\circ}%
%EndExpansion
}\right) & \cos \left( 64.\,\allowbreak 624\unit{%
%TCIMACRO{\U{b0}}%
%BeginExpansion
{{}^\circ}%
%EndExpansion
}\right) \sin \left( 108.\,\allowbreak 43\unit{%
%TCIMACRO{\U{b0}}%
%BeginExpansion
{{}^\circ}%
%EndExpansion
}\right) & -\sin \left( 64.\,\allowbreak 624\unit{%
%TCIMACRO{\U{b0}}%
%BeginExpansion
{{}^\circ}%
%EndExpansion
}\right) \\ 
-\sin \left( 108.\,\allowbreak 43\unit{%
%TCIMACRO{\U{b0}}%
%BeginExpansion
{{}^\circ}%
%EndExpansion
}\right) & \cos \left( 108.\,\allowbreak 43\unit{%
%TCIMACRO{\U{b0}}%
%BeginExpansion
{{}^\circ}%
%EndExpansion
}\right) & 0%
\end{array}%
\right] \left[ 
\begin{array}{c}
6 \\ 
1 \\ 
0%
\end{array}%
\right] \\
\left[ 
\begin{array}{c}
A_{r} \\ 
A_{\theta } \\ 
A_{\phi }%
\end{array}%
\right] &=&\left[ 
\begin{array}{c}
-0.856\,68 \\ 
-0.406\,34 \\ 
-6.\,\allowbreak 008\,4%
\end{array}%
\right] \allowbreak
\end{eqnarray*}

$\therefore \mathbf{A=}-0.856\,68\mathbf{a}_{r}-0.406\,34\mathbf{a}_{\theta
}-6.\,\allowbreak 008\,4\mathbf{a}_{\phi }$ (spherical coordinates)

$\therefore \mathbf{A=}6\mathbf{a}_{x}+\mathbf{a}_{y}=-0.948\,16\mathbf{a}%
_{\rho }+-6.\,\allowbreak 008\,4\,\mathbf{a}_{\phi }=-0.856\,68\mathbf{a}%
_{r}-0.406\,34\mathbf{a}_{\theta }-6.\,\allowbreak 008\,4\mathbf{a}_{\phi }$

Getting $\left\vert \mathbf{A}\right\vert $ for all coordinate system%
\begin{eqnarray*}
\left\vert \mathbf{A}\right\vert _{Cartesian} &=&\sqrt{%
A_{x}^{2}+A_{y}^{2}+A_{z}^{2}}=\sqrt{6^{2}+1^{2}+0}=6.\,\allowbreak 082\,8 \\
\left\vert \mathbf{A}\right\vert _{cylindrical} &=&\sqrt{A_{\rho
}^{2}+A_{\phi }^{2}+A_{z}^{2}}=\sqrt{\left( -0.948\,16\right) ^{2}+\left(
-6.\,\allowbreak 008\,4\right) ^{2}+0}=6.\,\allowbreak 082\,8 \\
\left\vert \mathbf{A}\right\vert _{spherical} &=&\sqrt{A_{r}^{2}+A_{\theta
}^{2}+A_{\phi }^{2}}=\sqrt{\left( -0.856\,68\right) ^{2}+\left(
-0.406\,34\right) ^{2}+\left( -6.\,\allowbreak 008\,4\right) ^{2}}%
=6.\,\allowbreak 082\,8
\end{eqnarray*}%
$\therefore $ the answers are correct.

\bigskip

\subsubsection{Practice Exercise 2.1.}

(a) Convert points $P\left( 1,3,5\right) ,$ $T\left( 0,-4,3\right) $, and $%
S\left( -3,-4,-10\right) $ from Cartesian to cylindrical and spherical
coordinates.

(b) Transform vector 
\[
\mathbf{Q=}\dfrac{\sqrt{x^{2}+y^{2}}\mathbf{a}_{x}}{\sqrt{x^{2}+y^{2}+z^{2}}}%
-\dfrac{yz\mathbf{a}_{z}}{\sqrt{x^{2}+y^{2}+z^{2}}} 
\]%
to cylindrical and spherical coordinates.

(c) Evaluate $\mathbf{Q}$ at $T$ in the three coordinate systems.

Solution:

(a) $P\left( 1,3,5\right) \longrightarrow P\left( \rho ,\phi ,z\right)
\longrightarrow P\left( r,\theta ,\phi \right) $

\begin{eqnarray*}
\rho &=&\sqrt{x^{2}+y^{2}}=\sqrt{1^{2}+3^{2}}=\sqrt{10}=3.\,\allowbreak
162\,3 \\
\phi &=&\arctan \left( \dfrac{y}{x}\right) =\arctan \left( \dfrac{3}{1}%
\right) =71.\,\allowbreak 562\unit{%
%TCIMACRO{\U{b0}}%
%BeginExpansion
{{}^\circ}%
%EndExpansion
} \\
z &=&5
\end{eqnarray*}

\begin{eqnarray*}
r &=&\sqrt{x^{2}+y^{2}+z^{2}}=\sqrt{1^{2}+3^{2}+5^{2}}=\sqrt{35}=\allowbreak
5.\,\allowbreak 916\,1 \\
\theta &=&\arctan \left( \dfrac{\sqrt{x^{2}+y^{2}}}{z}\right) =\arctan
\left( \dfrac{\sqrt{1^{2}+3^{2}}}{5}\right) =32.\,\allowbreak 311\unit{%
%TCIMACRO{\U{b0}}%
%BeginExpansion
{{}^\circ}%
%EndExpansion
} \\
\phi &=&\arctan \left( \dfrac{y}{x}\right) =\arctan \left( \dfrac{3}{1}%
\right) =71.\,\allowbreak 562\unit{%
%TCIMACRO{\U{b0}}%
%BeginExpansion
{{}^\circ}%
%EndExpansion
}
\end{eqnarray*}

$\therefore $ $P\left( 1,3,5\right) \longrightarrow P\left( 3.\,\allowbreak
162\,3,71.\,\allowbreak 562\unit{%
%TCIMACRO{\U{b0}}%
%BeginExpansion
{{}^\circ}%
%EndExpansion
},5\right) \longrightarrow P\left( 5.\,\allowbreak 916\,1,32.\,\allowbreak
311\unit{%
%TCIMACRO{\U{b0}}%
%BeginExpansion
{{}^\circ}%
%EndExpansion
},71.\,\allowbreak 562\unit{%
%TCIMACRO{\U{b0}}%
%BeginExpansion
{{}^\circ}%
%EndExpansion
}\right) $

$T\left( 0,-4,3\right) \longrightarrow T\left( \rho ,\phi ,z\right)
\longrightarrow T\left( r,\theta ,\phi \right) $ 
\begin{eqnarray*}
\rho &=&\sqrt{x^{2}+y^{2}}=\sqrt{0^{2}+\left( -4\right) ^{2}}=4 \\
\phi &=&\arctan \left( \dfrac{y}{x}\right) =\arctan \left( \dfrac{-4}{0}%
\right) =270\unit{%
%TCIMACRO{\U{b0}}%
%BeginExpansion
{{}^\circ}%
%EndExpansion
} \\
z &=&3
\end{eqnarray*}
\begin{eqnarray*}
r &=&\sqrt{x^{2}+y^{2}+z^{2}}=\sqrt{0^{2}+\left( -4\right) ^{2}+3^{2}}=5 \\
\theta &=&\arctan \left( \dfrac{\sqrt{x^{2}+y^{2}}}{z}\right) =\arctan
\left( \dfrac{\sqrt{0^{2}+\left( -4\right) ^{2}}}{3}\right)
=53.\,\allowbreak 13\unit{%
%TCIMACRO{\U{b0}}%
%BeginExpansion
{{}^\circ}%
%EndExpansion
} \\
\phi &=&\arctan \left( \dfrac{y}{x}\right) =\arctan \left( \dfrac{-4}{0}%
\right) =270\unit{%
%TCIMACRO{\U{b0}}%
%BeginExpansion
{{}^\circ}%
%EndExpansion
}
\end{eqnarray*}

$\therefore T\left( 0,-4,3\right) \longrightarrow T\left( 4,270\unit{%
%TCIMACRO{\U{b0}}%
%BeginExpansion
{{}^\circ}%
%EndExpansion
},3\right) \longrightarrow T\left( 5,53.\,\allowbreak 13\unit{%
%TCIMACRO{\U{b0}}%
%BeginExpansion
{{}^\circ}%
%EndExpansion
},270\unit{%
%TCIMACRO{\U{b0}}%
%BeginExpansion
{{}^\circ}%
%EndExpansion
}\right) $

$S\left( -3,-4,-10\right) \longrightarrow S\left( \rho ,\phi ,z\right)
\longrightarrow S\left( r,\theta ,\phi \right) $ 
\begin{eqnarray*}
\rho &=&\sqrt{x^{2}+y^{2}}=\sqrt{\left( -3\right) ^{2}+\left( -4\right) ^{2}}%
=5 \\
\phi &=&\arctan \left( \dfrac{y}{x}\right) =\arctan \left( \dfrac{-4}{-3}%
\right) =\left( 180\unit{%
%TCIMACRO{\U{b0}}%
%BeginExpansion
{{}^\circ}%
%EndExpansion
}+53.\,\allowbreak 13\unit{%
%TCIMACRO{\U{b0}}%
%BeginExpansion
{{}^\circ}%
%EndExpansion
}\right) =233.\,\allowbreak 13\unit{%
%TCIMACRO{\U{b0}}%
%BeginExpansion
{{}^\circ}%
%EndExpansion
} \\
z &=&-10
\end{eqnarray*}
\begin{eqnarray*}
r &=&\sqrt{x^{2}+y^{2}+z^{2}}=\sqrt{\left( -3\right) ^{2}+\left( -4\right)
^{2}+\left( -10\right) ^{2}}=5\sqrt{5}=\allowbreak 11.\,\allowbreak 18 \\
\theta &=&\arctan \left( \dfrac{\sqrt{x^{2}+y^{2}}}{z}\right) =\arctan
\left( \dfrac{\sqrt{\left( -3\right) ^{2}+\left( -4\right) ^{2}}}{-10}%
\right) =180\unit{%
%TCIMACRO{\U{b0}}%
%BeginExpansion
{{}^\circ}%
%EndExpansion
}-26.\,\allowbreak 565\unit{%
%TCIMACRO{\U{b0}}%
%BeginExpansion
{{}^\circ}%
%EndExpansion
}=153.\,\allowbreak 44\unit{%
%TCIMACRO{\U{b0}}%
%BeginExpansion
{{}^\circ}%
%EndExpansion
} \\
\phi &=&\arctan \left( \dfrac{y}{x}\right) =\arctan \left( \dfrac{-4}{-3}%
\right) =\left( 180\unit{%
%TCIMACRO{\U{b0}}%
%BeginExpansion
{{}^\circ}%
%EndExpansion
}+53.\,\allowbreak 13\unit{%
%TCIMACRO{\U{b0}}%
%BeginExpansion
{{}^\circ}%
%EndExpansion
}\right) =233.\,\allowbreak 13\unit{%
%TCIMACRO{\U{b0}}%
%BeginExpansion
{{}^\circ}%
%EndExpansion
}
\end{eqnarray*}

$\therefore S\left( -3,-4,-10\right) \longrightarrow S\left(
5,233.\,\allowbreak 13\unit{%
%TCIMACRO{\U{b0}}%
%BeginExpansion
{{}^\circ}%
%EndExpansion
},-10\right) \longrightarrow S\left( 11.\,\allowbreak 18,153.\,\allowbreak 44%
\unit{%
%TCIMACRO{\U{b0}}%
%BeginExpansion
{{}^\circ}%
%EndExpansion
},233.\,\allowbreak 13\unit{%
%TCIMACRO{\U{b0}}%
%BeginExpansion
{{}^\circ}%
%EndExpansion
}\right) $

(b) 
\[
\mathbf{Q=}\dfrac{\sqrt{x^{2}+y^{2}}\mathbf{a}_{x}}{\sqrt{x^{2}+y^{2}+z^{2}}}%
-\dfrac{yz\mathbf{a}_{z}}{\sqrt{x^{2}+y^{2}+z^{2}}} 
\]

let%
\begin{eqnarray*}
\rho &=&\sqrt{x^{2}+y^{2}} \\
y &=&\rho \sin \left( \phi \right)
\end{eqnarray*}

$\therefore $%
\[
\mathbf{Q=\dfrac{\sqrt{x^{2}+y^{2}}\mathbf{a}_{x}}{\sqrt{x^{2}+y^{2}+z^{2}}}-%
\dfrac{yz\mathbf{a}_{z}}{\sqrt{x^{2}+y^{2}+z^{2}}}=}\dfrac{\rho \mathbf{a}%
_{x}}{\sqrt{\rho ^{2}+z^{2}}}-\dfrac{\rho z\sin \left( \phi \right) \mathbf{a%
}_{z}}{\sqrt{\rho ^{2}+z^{2}}} 
\]

$\because \left( A_{x},A_{y},A_{z}\right) \longrightarrow \left( A_{\rho
},A_{\phi },A_{z}\right) $ 
\begin{eqnarray*}
\left[ 
\begin{array}{c}
A_{p} \\ 
A_{\phi } \\ 
A_{z}%
\end{array}%
\right] &=&\left[ 
\begin{array}{ccc}
\cos \left( \phi \right) & \sin \left( \phi \right) & 0 \\ 
-\sin \left( \phi \right) & \cos \left( \phi \right) & 0 \\ 
0 & 0 & 1%
\end{array}%
\right] \left[ 
\begin{array}{c}
A_{x} \\ 
A_{y} \\ 
A_{z}%
\end{array}%
\right] \\
&=&\left[ 
\begin{array}{ccc}
\cos \left( \phi \right) & \sin \left( \phi \right) & 0 \\ 
-\sin \left( \phi \right) & \cos \left( \phi \right) & 0 \\ 
0 & 0 & 1%
\end{array}%
\right] \left[ 
\begin{array}{c}
\dfrac{\rho }{\sqrt{\rho ^{2}+z^{2}}} \\ 
0 \\ 
-\dfrac{\rho z\sin \left( \phi \right) }{\sqrt{\rho ^{2}+z^{2}}}%
\end{array}%
\right] \\
&=&\left[ 
\begin{array}{c}
\rho \dfrac{\cos \phi }{\sqrt{\rho ^{2}+z^{2}}} \\ 
-\rho \dfrac{\sin \phi }{\sqrt{\rho ^{2}+z^{2}}} \\ 
-z\rho \dfrac{\sin \phi }{\sqrt{\rho ^{2}+z^{2}}}%
\end{array}%
\right] \allowbreak \\
\left[ 
\begin{array}{c}
A_{p} \\ 
A_{\phi } \\ 
A_{z}%
\end{array}%
\right] &=&\left( \dfrac{\rho }{\sqrt{\rho ^{2}+z^{2}}}\right) \left[ 
\begin{array}{c}
\cos \phi \\ 
-\sin \phi \\ 
-z\sin \phi%
\end{array}%
\right] \allowbreak
\end{eqnarray*}

$\therefore \mathbf{Q=}\dfrac{\sqrt{x^{2}+y^{2}}\mathbf{a}_{x}}{\sqrt{%
x^{2}+y^{2}+z^{2}}}-\dfrac{yz\mathbf{a}_{z}}{\sqrt{x^{2}+y^{2}+z^{2}}}%
=\left( \dfrac{\rho }{\sqrt{\rho ^{2}+z^{2}}}\right) \left[ \cos \left( \phi
\right) \mathbf{a}_{\rho }-\sin \left( \phi \right) \mathbf{a}_{\phi }-z\sin
\left( \phi \right) \mathbf{a}_{z}\right] $

\begin{eqnarray*}
y &=&r\sin \left( \theta \right) \sin \left( \phi \right) \\
z &=&r\cos \left( \theta \right) \\
r &=&\sqrt{x^{2}+y^{2}+z^{2}} \\
\sqrt{x^{2}+y^{2}} &=&\sqrt{\left( r\sin \left( \theta \right) \cos \left(
\phi \right) \right) ^{2}+\left( r\sin \left( \theta \right) \sin \left(
\phi \right) \right) ^{2}} \\
&=&r\sin \left( \theta \right) \sqrt{\cos ^{2}\left( \phi \right) +\sin
^{2}\left( \phi \right) } \\
\sqrt{x^{2}+y^{2}} &=&r\sin \left( \theta \right)
\end{eqnarray*}

$\therefore $

\begin{eqnarray*}
\mathbf{Q} &\mathbf{=}&\dfrac{\sqrt{x^{2}+y^{2}}\mathbf{a}_{x}}{\sqrt{%
x^{2}+y^{2}+z^{2}}}-\mathbf{\dfrac{yz\mathbf{a}_{z}}{\sqrt{x^{2}+y^{2}+z^{2}}%
}} \\
&=&\dfrac{r\sin \left( \theta \right) \mathbf{a}_{x}}{r}-\dfrac{\left[ r\sin
\left( \theta \right) \sin \left( \phi \right) \right] \left[ r\cos \left(
\theta \right) \right] \mathbf{a}_{z}}{r} \\
\mathbf{Q} &=&\sin \left( \theta \right) \mathbf{a}_{x}-r\sin \left( \theta
\right) \cos \left( \theta \right) \sin \left( \phi \right) \mathbf{a}_{z}
\end{eqnarray*}

$\left( A_{x},A_{y},A_{z}\right) \longrightarrow \left( A_{r},A_{\theta
},A_{\phi }\right) $ 
\begin{eqnarray*}
\left[ 
\begin{array}{c}
A_{r} \\ 
A_{\theta } \\ 
A_{\phi }%
\end{array}%
\right] &=&\left[ 
\begin{array}{ccc}
\cos \left( \phi \right) & \sin \left( \phi \right) & 0 \\ 
-\sin \left( \phi \right) & \cos \left( \phi \right) & 0 \\ 
0 & 0 & 1%
\end{array}%
\right] \left[ 
\begin{array}{c}
A_{x} \\ 
A_{y} \\ 
A_{z}%
\end{array}%
\right] \\
&=&\left[ 
\begin{array}{ccc}
\sin \left( \theta \right) \cos \left( \phi \right) & \sin \left( \theta
\right) \sin \left( \phi \right) & \cos \left( \theta \right) \\ 
\cos \left( \theta \right) \cos \left( \phi \right) & \cos \left( \theta
\right) \sin \left( \phi \right) & -\sin \left( \theta \right) \\ 
-\sin \left( \phi \right) & \cos \left( \phi \right) & 0%
\end{array}%
\right] \left[ 
\begin{array}{c}
\sin \left( \theta \right) \\ 
0 \\ 
-r\sin \left( \theta \right) \cos \left( \theta \right) \sin \left( \phi
\right)%
\end{array}%
\right] \\
&=&\left[ 
\begin{array}{c}
\sin ^{2}\theta \cos \phi -r\sin \theta \cos ^{2}\theta \sin \phi \\ 
\sin \theta \cos \theta \cos \phi +r\sin ^{2}\theta \cos \theta \sin \phi \\ 
-\sin \theta \sin \phi%
\end{array}%
\right] \allowbreak \\
\left[ 
\begin{array}{c}
A_{r} \\ 
A_{\theta } \\ 
A_{\phi }%
\end{array}%
\right] &=&\left[ 
\begin{array}{c}
\left( \sin \theta \right) \left( \sin \theta \cos \phi -r\cos ^{2}\theta
\sin \phi \right) \\ 
\left( \sin \theta \cos \theta \right) \left( \cos \phi +r\sin \theta \sin
\phi \right) \\ 
-\sin \theta \sin \phi%
\end{array}%
\right] \allowbreak
\end{eqnarray*}

$\therefore $ $\mathbf{Q=\dfrac{\sqrt{x^{2}+y^{2}}\mathbf{a}_{x}}{\sqrt{%
x^{2}+y^{2}+z^{2}}}-\dfrac{yz\mathbf{a}_{z}}{\sqrt{x^{2}+y^{2}+z^{2}}}=}%
\left( \sin \theta \right) \left( \sin \theta \cos \phi -r\cos ^{2}\theta
\sin \phi \right) \mathbf{a}_{r}+\left( \sin \theta \cos \theta \right)
\left( \cos \phi +r\sin \theta \sin \phi \right) \mathbf{a}_{\theta }-\sin
\theta \sin \phi \mathbf{a}_{\phi }$

(c) $\because T\left( 0,-4,3\right) \longrightarrow T\left( 4,270\unit{%
%TCIMACRO{\U{b0}}%
%BeginExpansion
{{}^\circ}%
%EndExpansion
},3\right) \longrightarrow T\left( 5,53.\,\allowbreak 13\unit{%
%TCIMACRO{\U{b0}}%
%BeginExpansion
{{}^\circ}%
%EndExpansion
},270\unit{%
%TCIMACRO{\U{b0}}%
%BeginExpansion
{{}^\circ}%
%EndExpansion
}\right) $%
\begin{eqnarray*}
\mathbf{Q} &\mathbf{=}&\dfrac{\sqrt{0^{2}+\left( -4\right) ^{2}}\mathbf{a}%
_{x}}{\sqrt{0^{2}+\left( -4\right) ^{2}+3^{2}}}-\dfrac{\left( -4\right)
\left( 3\right) \mathbf{a}_{z}}{\sqrt{0^{2}+\left( -4\right) ^{2}+3^{2}}} \\
\mathbf{Q} &=&0.8\mathbf{a}_{x}+2.\,\allowbreak 4\mathbf{a}_{z}
\end{eqnarray*}
\begin{eqnarray*}
\mathbf{Q} &\mathbf{=}&\left( \dfrac{4}{\sqrt{4^{2}+3^{2}}}\right) \left[
\cos \left( 270\unit{%
%TCIMACRO{\U{b0}}%
%BeginExpansion
{{}^\circ}%
%EndExpansion
}\right) \mathbf{a}_{\rho }-\sin \left( 270\unit{%
%TCIMACRO{\U{b0}}%
%BeginExpansion
{{}^\circ}%
%EndExpansion
}\right) \mathbf{a}_{\phi }-z\sin \left( 270\unit{%
%TCIMACRO{\U{b0}}%
%BeginExpansion
{{}^\circ}%
%EndExpansion
}\right) \mathbf{a}_{z}\right] \\
&=&\left( \dfrac{4}{5}\right) \left( \left( 0\right) \mathbf{a}_{\rho
}-\left( -1\right) \mathbf{a}_{\phi }-\left( 3\right) \left( -1\right) 
\mathbf{a}_{z}\right) \\
\mathbf{Q} &=&0.8\mathbf{a}_{\phi }+2.\,\allowbreak 4\mathbf{a}_{z}
\end{eqnarray*}%
\begin{eqnarray*}
\mathbf{Q} &\mathbf{=}&\left( \sin \left( 53.\,\allowbreak 13\unit{%
%TCIMACRO{\U{b0}}%
%BeginExpansion
{{}^\circ}%
%EndExpansion
}\right) \right) \left( \sin \left( 53.\,\allowbreak 13\unit{%
%TCIMACRO{\U{b0}}%
%BeginExpansion
{{}^\circ}%
%EndExpansion
}\right) \cos \left( 270\unit{%
%TCIMACRO{\U{b0}}%
%BeginExpansion
{{}^\circ}%
%EndExpansion
}\right) -\left( 5\right) \left( \cos \left( 53.\,\allowbreak 13\unit{%
%TCIMACRO{\U{b0}}%
%BeginExpansion
{{}^\circ}%
%EndExpansion
}\right) ^{2}\right) \sin \left( 270\unit{%
%TCIMACRO{\U{b0}}%
%BeginExpansion
{{}^\circ}%
%EndExpansion
}\right) \right) \mathbf{a}_{r} \\
&&+\left( \sin \left( 53.\,\allowbreak 13\unit{%
%TCIMACRO{\U{b0}}%
%BeginExpansion
{{}^\circ}%
%EndExpansion
}\right) \cos \left( 53.\,\allowbreak 13\unit{%
%TCIMACRO{\U{b0}}%
%BeginExpansion
{{}^\circ}%
%EndExpansion
}\right) \right) \left( \cos \left( 270\unit{%
%TCIMACRO{\U{b0}}%
%BeginExpansion
{{}^\circ}%
%EndExpansion
}\right) +\left( 5\right) \sin \left( 53.\,\allowbreak 13\unit{%
%TCIMACRO{\U{b0}}%
%BeginExpansion
{{}^\circ}%
%EndExpansion
}\right) \sin \left( 270\unit{%
%TCIMACRO{\U{b0}}%
%BeginExpansion
{{}^\circ}%
%EndExpansion
}\right) \right) \mathbf{a}_{\theta } \\
&&-\sin \left( 53.\,\allowbreak 13\unit{%
%TCIMACRO{\U{b0}}%
%BeginExpansion
{{}^\circ}%
%EndExpansion
}\right) \sin \left( 270\unit{%
%TCIMACRO{\U{b0}}%
%BeginExpansion
{{}^\circ}%
%EndExpansion
}\right) \mathbf{a}_{\phi } \\
&=&\left( 0.8\right) \left( 0.8\left( 0\right) -\left( 5\right) \left(
0.6\right) ^{2}\left( -1\right) \right) \mathbf{a}_{r}+\left( 0.8\left(
0.6\right) \right) \left( 0+\left( 5\right) \left( 0.8\right) \left(
-1\right) \right) \mathbf{a}_{\theta }-\left( 0.8\right) \left( -1\right) 
\mathbf{a}_{\phi } \\
\mathbf{Q} &=&1.\,\allowbreak 44\mathbf{a}_{r}-1.\,\allowbreak 92\mathbf{a}%
_{\theta }+0.8\mathbf{a}_{\phi }
\end{eqnarray*}

$\therefore $ $\mathbf{Q=}0.8\mathbf{a}_{x}+2.\,\allowbreak 4\mathbf{a}%
_{z}=0.8\mathbf{a}_{\phi }+2.\,\allowbreak 4\mathbf{a}_{z}=0.8\mathbf{a}%
_{\phi }+2.\,\allowbreak 4\mathbf{a}_{z}$

\bigskip

\subsubsection{Example 2.2.}

Express vector 
\[
\mathbf{B=}\dfrac{10}{r}\mathbf{a}_{r}+r\cos \left( \theta \right) \mathbf{a}%
_{\theta }+\mathbf{a}_{\phi }
\]%
in Cartesian and cylindrical coordinates. Find $\mathbf{B}\left(
-3,4,0\right) $ and $\mathbf{B}\left( 5,\dfrac{\pi }{2},-2\right) $.

Solution:

Using $\left( B_{r},B_{\theta },B_{\phi }\right) $ $\longrightarrow \left(
B_{x},B_{y},B_{z}\right) $ 
\begin{eqnarray*}
\left[ 
\begin{array}{c}
B_{x} \\ 
B_{y} \\ 
B_{z}%
\end{array}%
\right]  &=&\left[ 
\begin{array}{ccc}
\sin \left( \theta \right) \cos \left( \phi \right)  & \cos \left( \theta
\right) \cos \left( \phi \right)  & -\sin \left( \phi \right)  \\ 
\sin \left( \theta \right) \sin \left( \phi \right)  & \cos \left( \theta
\right) \sin \left( \phi \right)  & \cos \left( \phi \right)  \\ 
\cos \left( \theta \right)  & -\sin \left( \theta \right)  & 0%
\end{array}%
\right] \left[ 
\begin{array}{c}
B_{r} \\ 
B_{\theta } \\ 
B_{\phi }%
\end{array}%
\right]  \\
&=&\left[ 
\begin{array}{ccc}
\sin \left( \theta \right) \cos \left( \phi \right)  & \cos \left( \theta
\right) \cos \left( \phi \right)  & -\sin \left( \phi \right)  \\ 
\sin \left( \theta \right) \sin \left( \phi \right)  & \cos \left( \theta
\right) \sin \left( \phi \right)  & \cos \left( \phi \right)  \\ 
\cos \left( \theta \right)  & -\sin \left( \theta \right)  & 0%
\end{array}%
\right] \left[ 
\begin{array}{c}
\dfrac{10}{r} \\ 
r\cos \left( \theta \right)  \\ 
1%
\end{array}%
\right]  \\
\left[ 
\begin{array}{c}
B_{x} \\ 
B_{y} \\ 
B_{z}%
\end{array}%
\right]  &=&\left[ 
\begin{array}{c}
\dfrac{10}{r}\cos \phi \sin \theta +r\cos ^{2}\theta \cos \phi -\sin \phi 
\\ 
\dfrac{10}{r}\sin \theta \sin \phi +r\cos ^{2}\theta \sin \phi +\cos \phi 
\\ 
\dfrac{10}{r}\cos \theta 
\end{array}%
\right] 
\end{eqnarray*}

$\therefore $%
\begin{eqnarray*}
B_{x} &=&\dfrac{10}{r}\cos \phi \sin \theta +r\cos ^{2}\theta \cos \phi
-\sin \phi  \\
B_{y} &=&\dfrac{10}{r}\sin \theta \sin \phi +r\cos ^{2}\theta \sin \phi
+\cos \phi  \\
B_{z} &=&\dfrac{10}{r}\cos \theta -r\cos \theta \sin \theta 
\end{eqnarray*}


\end{document}
